%%%%%%%%%%%%%%%%%%%%%%%%%%%%%%%%%%%%%%%%%%%%%%%%%%%%%%%%%%%%%%%%%%%%%%
% LaTeX Template: Curriculum Vitae
%
% Source: http://www.howtotex.com/
% Feel free to distribute this template, but please keep the
% referal to HowToTeX.com.
% Date: July 2011
%
\documentclass[paper=a4,fontsize=11pt]{scrartcl} % KOMA-article class

\usepackage[english]{babel}
\usepackage{paralist}
\usepackage[utf8x]{inputenc}
\usepackage[protrusion=true,expansion=true]{microtype}
\usepackage{amsmath,amsfonts,amsthm}     % Math packages
\usepackage{graphicx}                    % Enable pdflatex
\usepackage[svgnames]{xcolor}            % Colors by their 'svgnames'
\usepackage{geometry}
	\textheight=700px                    % Saving trees ;-)
\usepackage[colorlinks=true, allcolors=cyan]{hyperref}
\usepackage{hang}
\usepackage{lmodern}

\setlength{\parindent}{0pt}

%\frenchspacing              % Better looking spacings after periods
\pagestyle{empty}           % No pagenumbers/headers/footers

%%% Custom sectioning (sectsty package)
%%% ------------------------------------------------------------
\usepackage{sectsty}

\sectionfont{%			            % Change font of \section command
	\usefont{OT1}{lmss}{b}{n}%		% bch-b-n: CharterBT-Bold font
	\sectionrule{0pt}{0pt}{-5pt}{2pt}}

%%% Macros
%%% ------------------------------------------------------------
\newlength{\spacebox}
\settowidth{\spacebox}{8888888888}			% Box to align text
\newcommand{\sepspace}{\vspace*{1em}}		% Vertical space macro

\newcommand{\MyName}[1]{ % Name
		\Huge \usefont{OT1}{lmss}{b}{n} \hfill #1
		\par \normalsize \normalfont}

\newcommand{\MySlogan}[1]{ % Slogan (optional)
		\large \usefont{OT1}{lmss}{m}{n} \textit{#1}
		\par \normalsize \normalfont}

\newcommand{\NewPart}[1]{\section*{\uppercase{#1}}}


\newcommand{\EducationEntry}[4]{
		\noindent{
			\textbf{#1} \textit{#3} \hfill      % Study
			\colorbox{Black}{%
				\parbox{6.5em}{%
					\hfill\color{White}\textbf{#2}
				}
			}
			\par  % Duration
			\hangindent=1em\hangafter=0 \small #4 % Description
			\hangindent=1em \normalsize \par
		}
}


\newcommand{\SkillsEntry}[2]{      % Same as \PersonalEntry
		\noindent\hangindent=2em\hangafter=0 % Indentation
		\parbox{\spacebox}{        % Box to align text
		\textit{#1}}			   % Entry name (birth, address, etc.)
		\hspace{1.5em} #2 \par}    % Entry value



\newcommand{\hang}[2]
{
\begin{labeledpar}
    {88888888}{\textit{#1}}{#2}
\end{labeledpar}
}
 \renewcommand{\familydefault}{\sfdefault}

%%% Begin Document
%%% ------------------------------------------------------------
\begin{document}
% you can upload a photo and include it here...
%\begin{wrapfigure}{l}{0.5\textwidth}
%	\vspace*{-2em}
%		\includegraphics[width=0.15\textwidth]{photo}
%\end{wrapfigure}

\MyName{Fred Cook}
\MySlogan{fred@fredcook.co.uk\\\href{https://www.github.com/fuverdred}{github.com/fuverdred}}


%\sepspace
%%% Education
%%% ------------------------------------------------------------
\NewPart{Education}{}
\EducationEntry{PhD Physics}{2016-Present}{University of Bristol}
{
    Thesis submitted December 2020, awaiting viva date. \newline
    Title: \textit{Development of Apparatus for Ice Nucleation Studies.}\\
    \\
    The fundamentals of what makes a good ice nucleator remain poorly understood at the nanoscale. In my PhD I developed three experimental methods:
    \begin{itemize}
        \setlength\itemsep{0em}
        \item A novel way of automating a standard experimental technique (\href{https://amt.copernicus.org/articles/13/2785/2020/}{published}).
        \item An updated version of an automated lag time apparatus (\href{https://aip.scitation.org/doi/10.1063/1.1145586}{ALTA}) for ice nucleation studies.
        \item An environmental chamber for freezing acoustically levitated water droplets.
    \end{itemize}
    Some relevant highlights of my PhD work include:
    \begin{itemize}
        \setlength\itemsep{0em}
        \item A program for detecting freezing droplets from a series of images, including tracking the movement of the droplets, written in Python using OpenCV.
        \item Reverse engineering the instruction set for a picolitre droplet printer, allowing a custom labVIEW program integrated with an X-Y translation stage to be written.
        \item Python scripts for cleaning, analysing and graphing data using standard scientific libraries (NumPy, SciPy and Matplotlib).
        \item Programmed microcontrollers (Arduino and pyBoard) to read peripherals and control experiments.
    \end{itemize}
}

\EducationEntry{MSci Physics}{2012-2016}{University of Bristol}
{
    First class Honours
}
%\sepspace

\EducationEntry{Alleyn's School}{2005-2012}{}
{
    A-levels: Physics A*, Maths A*, Economics A (AS-level Politics A) \\
    GCSEs: 6A*, 3A, 1B
}
%\newline
%GCSE/IGCSE: 6$\times$A*, 3$\times$A, 1$\times$B}

%%% Skills
%%% ------------------------------------------------------------
\NewPart{Software Development Skills}{}

\hang{Python 3}{
    Four years of experience. \\
    Well versed in the standard scientific libraries: \textbf{NumPy}, \textbf{SciPy}, \textbf{Matplotlib}. \\
		Personal projects include web-scraping scripts, a \href{https://github.com/fuverdred/Crossword-Filler}{tool for creating themed crosswords} and a \href{https://github.com/fuverdred/pH-Monitor}{device for monitoring and controlling the pH of soil}.
}

\hang{Misc.}{
    Experience programming in C, knowledge of git and \LaTeX
}

%%% Work experience
%%% ------------------------------------------------------------
\NewPart{Publications}{}
\begin{itemize}
    \item Cook et al., \href{https://amt.copernicus.org/articles/13/2785/2020/}{\textit{A pyroelectric thermal sensor for automated ice nucleation detection.}} (2020) Atmos. Meas. Tech. Disc. 13, 2785–2795
    \item Cook et al., \textit{An updated automated lag-time apparatus for ice nucleation studies.} Awaiting submission.
\end{itemize}

\end{document}